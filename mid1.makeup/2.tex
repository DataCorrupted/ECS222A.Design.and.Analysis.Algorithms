
\section*{Q2}

\begin{lstlisting}
void foo(Array A) {
    n=A.size();
    print "n";
    if (n<=1) return;
    foo(A[1...n/y]);
}
        
\end{lstlisting}

Analyze the above code in the case that the data is being passed by value not reference.
For example, B=[1,2,3,4,....100] call foo(B); makes a full copy of B into A.

\subsection*{Solution}

$$T(n) = T(n/y) + n/y$$

Using master theorem, $a = 1, b = y, f(n) = n/y = O(n)$, $c = \log_b^a < 1$

Thus $$T(n) = \Theta(f(n)) = O(n)$$