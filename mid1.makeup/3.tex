
\section*{Q3}

Rod cutting problem with cost per cut:
Assume that the rod cutter you are using charges you a different price for different size cuts.
Therefore:
Given:
n, the length of the rod

P[1..n] s.t. P[i] is the price you can sell rod length ii

C[1..n] s.t. C[i] is the cost of cutting rod length i

Output:  A plan to cut up the rod and sell the pieces in
a way that maximizes the total profit (amount of money you get minus the costs for cutting).

\subsection*{Solution}

The problem is ambigious. ``C[i] is cost of cutting rod length i'' has two interpretations:

\begin{enumerate}
    \item ``C[i] is cost of cutting rod AT length i''
    \item ``C[i] is cost of cutting rod FOR length i''
\end{enumerate}

If using the second interpretation, you can use $P-C$ as your new price and it's the same as rod cutting problem.
Therefore, here we assume interpretation one.

\textbf{Recurrence}

We can solve this by asking ``for all k, if we cut at k, the max value I can get by rod k...n''.

\begin{lstlisting}
    int rod_cut(size_t n){
        int max_val = P[n];
        for (int i=1; i<n-1; i++) {
            int cur_val = rod_cut(i) + P[n-i] - C[i];
            if (cur_val > max_val) {
                max_val = cur_val;
            }
        }
        return max_val;
    }
\end{lstlisting}

\textbf{DP}

Use \text{opt[i]} for ``Given rod only has length i, the max profit we are going to have.''

Then $$\text{opt[i]} = \max(P[i], \max_{\forall 0 < k < i}(opt[i-k] + P[k] - C[i-k]))$$