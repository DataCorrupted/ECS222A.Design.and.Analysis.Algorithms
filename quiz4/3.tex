
\section*{Q3}

\begin{enumerate}
    \item Given $LCP(Suff_4, Suff_{y+10})=10$, what is the $LCP(Suff_5, Suff_{y+11})$, explain your answer.
    \item Given $LCP(Suff_4, Suff_{17})=10$, what is the $LCP(Suff_3, Suff_{16})$, explain what your answer could be (maybe several options).
    \item Draw a suffix subtree for suffix i,j,k such that
          \begin{align*}
              suff_i & =string1.string2.string3 \\
              suff_j & =string1.string3         \\
              suff_k & =string1.string2.string1 \\
          \end{align*}

\end{enumerate}

\subsection*{Solution}

\begin{enumerate}
    \item 9.

          For similicity, let's use $1$ to denote characters that are the same, $0$ for different, $x$ other ones(May be the same, or different, we just don't know).
          $LCP(S_4, S_{17}) = 10$ implies that the string should look like:

          \begin{lstlisting}
               ...4 ...8 ...c ...f 
        0x0    xxx1 1111 1111 1xxx 
        0x1    1111 1111 11xx xxxx
          \end{lstlisting}

          Therefore, $LCP(S_5, S_{18}) = 9$

    \item Possibilities: 0 or 11.

          \begin{enumerate}
              \item 0. In this case, the 3rd and 16-th character is different, i.e.
                    \begin{lstlisting}
                ...4 ...8 ...c ...f 
         0x0    xx01 1111 1111 1xx0 
         0x1    1111 1111 11xx xxxx
                    \end{lstlisting}
              \item 11. In this case, the 3rd and 16-th character is in fact the same.
              \item 0. In this case, the 3rd and 16-th character is different, i.e.
                    \begin{lstlisting}
          ...4 ...8 ...c ...f 
   0x0    xx11 1111 1111 1xx1 
   0x1    1111 1111 11xx xxxx
                    \end{lstlisting}
          \end{enumerate}

    \item \begin{figure}
              \centering
              \begin{forest}
                  for tree={circle,draw, l sep=20pt, s sep=30pt}
                  [-
                  [-, edge label={node[midway, above left] {str1}}
                      [-, edge label={node[midway, left] {str2}}
                              [k,edge label={node[midway, left] {str1}}]
                              [i,edge label={node[midway, right] {str3}}]
                      ]
                      [j,edge label={node[midway, right] {str3}}]
                  ]
                  ]
              \end{forest}
          \end{figure}
\end{enumerate}
