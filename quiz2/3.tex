
\section*{Q3}

In the "pretty printing problem"(Kleinberg and Tardos Chapter 6, Problem 6) assume we are required that every line must have an even number of words.
(you may assume n is divisible by 2 otherwise there is no solution)

\begin{enumerate}
    \item Create a recurrence for the optimal solution.
    \item Create a dynamic program and analyze its runtime.
\end{enumerate}

\subsection*{Solution}

\begin{enumerate}
    \item Brute force.

          We use partial sums to accelate calculation. $P(j) = P(j-1) + w_j$, thus the sum of characters from $i$ to $j$ is $P(j) - P(i)$.
          \begin{lstlisting}[language = C]
// Tests if there will be a line that has more characters than L
bool is_valid(int k) {
    ...
}

int get_min_cost() {
    // Calculate P(k);

    // Brute force.
    int min_cost = 0x7fffffff;
    for (int k=1; k<n; k++){
        if (!is_valid(k)) {
            continue;
        }
        int cost = 0;

        // For each line, sum up the cost.

        if (cost < min_cost) {
            min_cost = cost;
        }
    }
    return min_cost;
}
    \end{lstlisting}
    \item DP solution.

          Let's assume $opt(k, j)$ being ``Suppose each line has $k$ words, we are working on $j$-th word''.
          Therefore, there are two implied border conditions:
          \begin{itemize}
              \item $\forall k, opt(k, 0) = L^2$, i.e. there is only one line with nothing in it.
              \item $\forall j, opt(0, j) = \infty$, i.e. we put nothing in each line, and thus it takes infinity lines, which is not valid.
          \end{itemize}

          Now we can derive a transfer function:

          \begin{minipage}{\linewidth}
              \begin{lstlisting}[language=c]
if (j % k == 0) {   // We have to start a new line or change k.
    opt[k, j] = min (
        // Start a new line.
        opt[k, j-1] + (*@ $(L-w_j)^2$ @*),     

        // Try a smaller k.
        opt[k-1, j],                            
    )
} else if (is_valid(k)) { // We can attach word j into the last line if there is enough space.
    opt[k, j] = min (
        // Suppose t is the index of the starting word.
        opt[k, j-1] - (*@ $S_{t, j-1}^2 + S_{t, j}^2$ @*), 

        // Try a smaller k.
        opt[k-1, j],                            
    )
} else {
    // k is too large to fit into j, try smaller one.
    opt[k, j] = opt[k-1, j];                   
}
\end{lstlisting}
          \end{minipage}
\end{enumerate}