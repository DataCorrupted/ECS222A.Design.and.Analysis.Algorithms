
\section*{Q4}

\begin{lstlisting}[language=C]
int foo(int A[]){
    n=A.size
    if (n<4) return 1;
    for(i=2 to n){
        call temp(A[1..i/2]);
    }

    int x = foo(A[1 ..n/y]) + foo(A[n/y ..2n/y]);
    return foo(A[1 ..n/y]) + x;
}
\end{lstlisting}

Given that temp has an asymptotic tight bound of $\Theta(n)$
\begin{enumerate}
    \item Provide the recurrence of the code above.
    \item Prove that the recurrence above $O(n^y)$ asymptotic upper bound using subsitution.
\end{enumerate}

\subsection*{Solution}

\textbf{Note:} From program analysis perspective, two function calls \lstinline{foo(A[1 ..n/y])} in line 8 and 9 will not be repeated.
The compiler will most likely to cache the result and use it without actually executing the second function call.
However, here we assume that every function call is executed with no optimization.


\begin{enumerate}
    \item $$T(n) = 3T(n/y) + n\Theta(n)$$ $$T(k) = 1, \text{where } k = 0, 1, 2, 3$$
    \item \begin{proof}
              Let's proof by induction.
              First assume $T(n) = O(n^y)$, i.e. $T(n) \le n^y$.

              Base case. When $n = 0, 1, 2, 3$, $T(n) = 1 \le n^y$.

              Induction case. Suppose $T(n/y) \le (n/y)^y$, then we have:

              \begin{align*}
                  T(n) & =   3T(n/y) + n\Theta(n) \\
                       & \le 3(n/y)^y + n^2       \\
                       & < n^y
              \end{align*}

              Therefore, $T(n) = O(n^y)$
          \end{proof}
\end{enumerate}