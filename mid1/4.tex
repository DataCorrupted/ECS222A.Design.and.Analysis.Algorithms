
\section*{Q4}

\begin{enumerate}
    \item Find the max flow using FF algorithm.

          \begin{figure}[H]
              \centering
              \begin{tikzpicture}[auto]
                  \tikzstyle{every node} = [align=center, draw, circle, node distance = 2cm]
                  \node(s){s};
                  \node[above right of=s](a){a};
                  \node[below right of=s](c){c};
                  \node[right of=a](b){b};
                  \node[right of=c](d){d};
                  \node[below right of=b](t){t};

                  \path (s) edge[cfgedge] node[draw=none, above](){11} (a);
                  \path (s) edge[cfgedge] node[draw=none, below](){12} (c);
                  \path (c) edge[cfgedge] node[draw=none](){1} (a);
                  \path (a) edge[cfgedge] node[draw=none](){12} (b);
                  \path (c) edge[cfgedge] node[draw=none](){11} (d);
                  \path (d) edge[cfgedge] node[draw=none](){7} (b);
                  \path (b) edge[cfgedge] node[draw=none, above](){19} (t);
                  \path (d) edge[cfgedge] node[draw=none, below](){14} (t);
              \end{tikzpicture}
          \end{figure}

          Show your work.
    \item If we decreased the capacity of the edge $(d,b)$ would this decrease the max flow?
    \item What is the min capacity cut of this network?
          What are the edges that cross the cut that make up this capacity?
\end{enumerate}

\subsection*{Solution}

\begin{enumerate}
    \item (I'm not sure if I have time to use tikz to draw all residual network...)

          \begin{enumerate}
              \item Initial residual graph $G_f$ is the same as $G$.
                    As shown in \autoref{fig:1}.
                    \begin{figure}[H]
                        \centering
                        \begin{tikzpicture}[auto]
                            \tikzstyle{every node} = [align=center, draw, circle, node distance = 2cm]
                            \node(s){s};
                            \node[above right of=s](a){a};
                            \node[below right of=s](c){c};
                            \node[right of=a](b){b};
                            \node[right of=c](d){d};
                            \node[below right of=b](t){t};

                            \path (s) edge[cfgedge] node[draw=none, above](){11} (a);
                            \path (s) edge[cfgedge] node[draw=none, below](){12} (c);
                            \path (c) edge[cfgedge] node[draw=none](){1} (a);
                            \path (a) edge[cfgedge] node[draw=none](){12} (b);
                            \path (c) edge[cfgedge] node[draw=none](){11} (d);
                            \path (d) edge[cfgedge] node[draw=none](){7} (b);
                            \path (b) edge[cfgedge] node[draw=none, above](){19} (t);
                            \path (d) edge[cfgedge] node[draw=none, below](){14} (t);
                        \end{tikzpicture}
                        \label{fig:1}
                        \caption{Initial graph.}
                    \end{figure}

              \item Find path $s \rightarrow a \rightarrow b \rightarrow t$ in residual network, min flow in this path is 11.
                    As shown in \autoref{fig:2}.
                    \begin{figure}[H]
                        \centering
                        \begin{tikzpicture}[auto]
                            \tikzstyle{every node} = [align=center, draw, circle, node distance = 2cm]
                            \node(s){s};
                            \node[above right of=s](a){a};
                            \node[below right of=s](c){c};
                            \node[right of=a](b){b};
                            \node[right of=c](d){d};
                            \node[below right of=b](t){t};

                            \path (a) edge[thickcfgedge] node[draw=none, above](){11} (s);
                            \path (s) edge[cfgedge] node[draw=none, below](){12} (c);
                            \path (c) edge[cfgedge] node[draw=none](){1} (a);
                            \path (a) edge[thickcfgedge, bend left] node[draw=none](){1} (b);
                            \path (b) edge[thickcfgedge, bend left] node[draw=none](){11} (a);
                            \path (c) edge[cfgedge] node[draw=none](){11} (d);
                            \path (d) edge[cfgedge] node[draw=none](){7} (b);
                            \path (t) edge[thickcfgedge, bend left] node[draw=none, above](){11} (b);
                            \path (b) edge[thickcfgedge, bend left] node[draw=none, above](){8} (t);
                            \path (d) edge[cfgedge] node[draw=none, below](){14} (t);
                        \end{tikzpicture}
                        \label{fig:2}
                        \caption{Residual graph after we found path $s \rightarrow a \rightarrow b \rightarrow t$.}

                    \end{figure}

              \item Find path $s \rightarrow c \rightarrow a \rightarrow b \rightarrow t$ in residual network, min flow in this path is 1.
                    As shown in \autoref{fig:3}.
                    \begin{figure}[H]
                        \centering
                        \begin{tikzpicture}[auto]
                            \tikzstyle{every node} = [align=center, draw, circle, node distance = 2cm]
                            \node(s){s};
                            \node[above right of=s](a){a};
                            \node[below right of=s](c){c};
                            \node[right of=a](b){b};
                            \node[right of=c](d){d};
                            \node[below right of=b](t){t};

                            \path (a) edge[cfgedge] node[draw=none, above](){11} (s);
                            \path (s) edge[thickcfgedge, bend left] node[draw=none, below](){11} (c);
                            \path (c) edge[thickcfgedge, bend left] node[draw=none, below](){1} (s);
                            \path (a) edge[thickcfgedge] node[draw=none](){1} (c);
                            \path (b) edge[thickcfgedge] node[draw=none](){12} (a);
                            \path (c) edge[cfgedge] node[draw=none](){11} (d);
                            \path (d) edge[cfgedge] node[draw=none](){7} (b);
                            \path (t) edge[thickcfgedge, bend left] node[draw=none, above](){12} (b);
                            \path (b) edge[thickcfgedge, bend left] node[draw=none, above](){7} (t);
                            \path (d) edge[cfgedge] node[draw=none, below](){14} (t);
                        \end{tikzpicture}
                        \label{fig:3}
                        \caption{Residual graph after we found path $s \rightarrow c \rightarrow a \rightarrow b \rightarrow t$.}
                    \end{figure}

              \item Find path $s \rightarrow c \rightarrow d \rightarrow t$ in residual network, min flow in this path is 11.
                    As shown in \autoref{fig:4}.
                    \begin{figure}[H]
                        \centering
                        \begin{tikzpicture}[auto]
                            \tikzstyle{every node} = [align=center, draw, circle, node distance = 2cm]
                            \node(s){s};
                            \node[above right of=s](a){a};
                            \node[below right of=s](c){c};
                            \node[right of=a](b){b};
                            \node[right of=c](d){d};
                            \node[below right of=b](t){t};

                            \path (a) edge[cfgedge] node[draw=none, above](){11} (s);
                            \path (s) edge[thickcfgedge, bend left] node[draw=none, below](){7} (c);
                            \path (c) edge[thickcfgedge, bend left] node[draw=none, below](){5} (s);
                            \path (a) edge[cfgedge] node[draw=none](){1} (c);
                            \path (b) edge[cfgedge] node[draw=none](){12} (a);
                            \path (c) edge[thickcfgedge, bend left] node[draw=none](){7} (d);
                            \path (d) edge[thickcfgedge, bend left] node[draw=none](){4} (c);
                            \path (d) edge[cfgedge] node[draw=none](){7} (b);
                            \path (t) edge[cfgedge, bend left] node[draw=none, above](){12} (b);
                            \path (b) edge[cfgedge, bend left] node[draw=none, above](){7} (t);
                            \path (t) edge[thickcfgedge] node[draw=none, below](){4} (d);
                        \end{tikzpicture}
                        \label{fig:4}
                        \caption{Residual graph after we found path $s \rightarrow c \rightarrow d \rightarrow t$}
                    \end{figure}

              \item Find path $s \rightarrow c \rightarrow d \rightarrow b \rightarrow t$ in residual network, min flow in this path is 7.
                    As shown in \autoref{fig:5}.
                    \begin{figure}[H]
                        \centering
                        \begin{tikzpicture}[auto]
                            \tikzstyle{every node} = [align=center, draw, circle, node distance = 2cm]
                            \node(s){s};
                            \node[above right of=s](a){a};
                            \node[below right of=s](c){c};
                            \node[right of=a](b){b};
                            \node[right of=c](d){d};
                            \node[below right of=b](t){t};

                            \path (a) edge[cfgedge] node[draw=none, above](){11} (s);
                            \path (c) edge[thickcfgedge] node[draw=none, below](){12} (s);
                            \path (a) edge[cfgedge] node[draw=none](){1} (c);
                            \path (b) edge[cfgedge] node[draw=none](){12} (a);
                            \path (d) edge[thickcfgedge] node[draw=none](){11} (c);
                            \path (b) edge[thickcfgedge] node[draw=none](){7} (d);
                            \path (t) edge[thickcfgedge] node[draw=none, above](){19} (b);
                            \path (t) edge[cfgedge] node[draw=none, below](){14} (d);
                        \end{tikzpicture}
                        \label{fig:5}
                        \caption{Residual graph after we found path $s \rightarrow c \rightarrow d \rightarrow b \rightarrow t$.}
                    \end{figure}

              \item $s$ to $t$ is disconnected in residual network, the algorithm is finished.
          \end{enumerate}

          Therefore, the max flow is 23.

    \item Yes, as there is a path that relies on $(d, b)$, i.e. it's the min flow in that path.
          Reducing $(d, b)$ would decrease the flow of the whole network.

    \item Since min cut is max flow, it is also 23.

          \begin{figure}[H]
              \centering
              \begin{tikzpicture}[auto]
                  \tikzstyle{every node} = [align=center, draw, circle, node distance = 2cm]
                  \node(s){s};
                  \node[above right of=s](a){a};
                  \node[below right of=s](c){c};
                  \node[right of=a](b){b};
                  \node[right of=c](d){d};
                  \node[below right of=b](t){t};

                  \path (a) edge[cfgedge] node[draw=none, above](){11} (s);
                  \path (c) edge[cfgedge] node[draw=none, below](){12} (s);
                  \path (a) edge[cfgedge] node[draw=none](){1} (c);
                  \path (b) edge[cfgedge] node[draw=none](){12} (a);
                  \path (d) edge[cfgedge] node[draw=none](){11} (c);
                  \path (b) edge[cfgedge] node[draw=none](){7} (d);
                  \path (t) edge[cfgedge] node[draw=none, above](){19} (b);
                  \path (t) edge[cfgedge] node[draw=none, below](){14} (d);
              \end{tikzpicture}
              \caption{Final residual graph}
          \end{figure}

          In the final residual graph, the connected components of $s$ is $\{s\}$.
          Thus the min cut would be all capacity out of $\{s\}$, i.e. $c(s, a) + c(s, c) = 23$.
\end{enumerate}